\documentclass[a4paper,12pt]{scrartcl}
\usepackage[utf8]{inputenc}
\usepackage[T1]{fontenc}
\usepackage[english,ngerman]{babel}
\usepackage{graphicx}
\usepackage{caption}
\usepackage{amsmath}
\usepackage{lmodern}
\usepackage{textcomp}
\usepackage[hidelinks]{hyperref}

%\selectlanguage{german}

\hyphenation{di-gi-tal vi-su-ali-sie-ren}

%Seitenabstände
\usepackage[top=2.5cm, bottom=2.0cm, left=2.5cm, right=2.5cm]{geometry}

\title{{\fontsize{30}{20}\selectfont Generieren und Darstellung von verschiedenen Galaxienmodellen}}
\subtitle{Praktikumsbericht}

\author{ Emile Hansmaennel \footnote{\url{ emile.hansmaennel@gmail.com }}}
\date{\today}
\setlength{\footskip}{30pt}

\begin{document}
 
 \maketitle 
 
 \begin{figure}[h]
  \centering
  \includegraphics[height=40mm]{figs/hd_logo_standard_16cm_rgb.png}\hfill
  \includegraphics[height=40mm]{figs/zah-logo-bgtrans-eps-converted-to.pdf}
  \captionsetup{labelformat=empty}
  \caption{}
 \end{figure}
  
 \thispagestyle{empty}
 \clearpage
 \newpage
 \setcounter{page}{1}
 
 \tableofcontents
 \newpage
 
 \section{Einleitung} \label{Einleitung}
\subsection{Galaxien, Dark Matter bla bla}

 \newpage
 
 \section{Generieren einer elliptischen Galaxie} \label{Hauptteil}
\subsection{NFW Stuff}
\subsection{}

 \newpage
 
 \section{Ergebnisse} \label{ergebnisse}
\subsection{Eigenes Zeug}
\subsection{Thales Zeug}

 \newpage
 
 \section{Quellen und Hilfen} \label{quellen}

 
\end{document}
