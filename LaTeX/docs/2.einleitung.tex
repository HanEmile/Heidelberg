Die Bahn eines Satelliten zu berechnen, diese zu visualisieren und anschließend eine eventuelle
Kollision mit einem anderem Satelliten zu erschließen war der anfängliche Plan.
Um ein Programm herzustellen, das zu all diesem imstande ist, gibt es jedoch eine Vorraussetzung:
Der jenige der das Programm erstellen will, muss eine Programmiersprache beherrschen.
\par
Diese Vorraussetzung erfüllte ich jedoch nicht, was mich jedoch nicht davon abhielt mir eine
Programmiersprache aus der Vielzahl verschiedener Möglichkeiten auszuwählen und das Projekt damit zu durchlaufen.
Meine Wahl fiel auf Python, da ich eine strukturierte Programmiersprache gesucht habe,
welche damit auch zu komplexeren Aufgaben imstande ist.
Ein weiteres Argument für Python war, dass ich die Komplexen Zusammenhänge visualisieren wollte. Ich habe mich direkt
für das 3D-Programm Blender entschieden, da ich mich in diesem Programm bereits sehr gut
auskenne, es unglaublich flexibel ist und es schon eine Python-Integration besitzt.
Somit konnte ich mithilfe von Python Blender steuern und die Bahnen der Satelliten und die
entsprechenden Satelliten einfügen. 
\par
Um die Satelliten darzustellen brachte ich Real-Daten, jedoch wusste ich
nicht woher ich diese bekommen sollte. Nach einer intensiven Internet-Recherche fand ich eine
Quelle, welche die Daten aller Satelliten im Orbit im sogenanntem Two-Line-Element (dt. Zwei Zeilen Element) Format anbot:
(http://celestrak.com/).
\par
Mit der Zeit kamen aber auch weitere Probleme hinzu, welche sich nicht so einfach lösen ließen und es
mussten Umwege und Alternativwege genutzt werden, um an das Ziel zu kommen, welches ich mit
Freude am Ende erreichte.



